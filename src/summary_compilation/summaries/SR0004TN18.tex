\section{ SR0004TN18 }


\subsection{Meta}

    \textbf{Title:}
    Surgery case scheduling in a multistage operating room department: A literature review

    \begin{table}[H]
        \centering
        \begin{tabular}{|c|c|c|c|c|c|c|c|c|}
            \hline
                \textbf{Rank} & \textbf{Grasp} & \textbf{Grade} & \textbf{Type} & \textbf{Outcome} & \textbf{Domain} & \textbf{COV19} & \textbf{CoI} & \textbf{DB} \\
            \hline
                5 & 80\% & F & A & P & S & No & - & - \\
            \hline
        \end{tabular}
        \caption{Reference's metadata}
        \label{tab:SR0004TN18}
    \end{table}

\subsection{Summary}
    Marwa Khalfalli \cite{x104} demonstrated the work with an unclear structure and objectives. There are no supportive visuals in the text. The study is hard to read and comprehend due to the ever-changing narrative. The author presents an unknown principle of two-stage scheduling: the first stage is a surgery case allocation, and the second is sequential scheduling. I \textbf{do not recommend} using this paper as a guide for research.

\subsection{Reading}
    \textbf{Abstract:}
    The operating theatre scheduling is a complex problem which involves medical personnel and other resources. The surgery case scheduling in a multistage operating room department is presented in the work.

    \textbf{Page 1:}
    OR management is one of the most important spheres in a hospital. Two-step scheduling process includes allocation and sequencing of ORs. Two steps are considered as separate combinatorial problems. OR department consists of Public Health Unit (PHU), OR, and PACU. There are three operative phaces. 
    
    \textbf{Page 2:}
    Intraoperative phace is the core of the surgery operation which requires multiple resources. In post-operative phace, the patient is transfert either to PACU or ICU. PACU may become a bottleneck of the surgery operation flow. ICU is closelly connected to OR utilisation and patient satisfaction level. Further an example in the case study was given and the integration scheduling introduced. More of the literature review summaries in the following paragraphs.
    
    \textbf{Page 3:}
    In the left half of the page the author dives vague details regarding the two-stage operating room department particualrly the proposed problem description. The right half eliberates more on the second stage of the scheduling process and presents more summaries of the existing studies.
    
    \textbf{Page 4:}
    Many not coherent summaries of the different scheduling models.
    
    \textbf{Page 5:}
    Introducing studies in the multi-objective scheduling.
    
    \textbf{Concusions:}
    There are three concluding ideas: more considerations should be put into downstream and upstream untis; general thoughts on two the most important critarias such as overtime and utilisation; and highlights some new design. (what new design?)