\section{ SR02CN23 }


\subsection{Meta}

    \textbf{Title:}
    A Literature Review of Service Capacity Planning for Medical Technology Department

    \begin{table}[H]
        \centering
        \begin{tabular}{|c|c|c|c|c|c|c|c|c|}
            \hline
                \textbf{Rank} & \textbf{Grasp} & \textbf{Grade} & \textbf{Type} & \textbf{Outcome} & \textbf{Domain} & \textbf{COV19} & \textbf{CoI} & \textbf{DB} \\
            \hline
                2 & 95\% & D & A & P & B & No & ?? & No \\
            \hline
        \end{tabular}
        \caption{Reference's metadata}
        \label{tab:SR02CN23}
    \end{table}

\subsection{Summary}
    Hongying Fei and Yiming Kang\cite{x243} produced a general overview of the existing literature regarding service capacity planning for the medical technology department. The review needs a more substantial structure. The authors categorised the literature by patient type, attendance, and uncertainty. The summaries of the reviewed literature are presented. The conducted study has little value for newcomers who want to gain a general understanding of existing research.

\subsection{Notes}
    \begin{itemize}
        \item Projection method?
    \end{itemize}


\subsection{Reading}
    \textbf{Abstract:}
    The medical services play crusial role in the quality of healthcare system. The current review overviews, structures, and summarises latest research in the topic of service capacity planning.
    
    \textbf{Objectives:}
    To identify and structure the most recent studies in the field of healthcare service capacity planning.
    
    \textbf{Page 1 (Introduction):}
    There is more demand for healthcare services then capacity. The optimisation of healthcare capacities requires an efficient management of the expancive and valuable resources (surgeons, nurses, and modern medical equipment). Two stages form a healthcare service capacity planning: resource allocation and capacity design. Medical Technology Department is a significant part of healthcare. The authors review and analyse the existing literature and then draw conclusions and suggestions from the conducted review.  
    
    \textbf{Page 2 (Essentials for MT department):}
    The most valuable aspects of MT department are: patient beds, medical personnel, operating rooms, and the medical equipment. The demand on MT services is drastically increased.

    \textbf{Page 3 (Service Capacity Planning):}
    \underline{Demand}. The demand is an important factor in evaluation healthcare needs. Here some discussion and possible demand estimation methods from the literature are presented. 

    \textbf{Page 4 (Service Capacity Planning):}
    \underline{Patient Attendance}. List of papers. Patient demand and no-show prediction. Unsertainty. \underline{Uncertain Service Time}. Stochastic models for patients waiting time prediction and uncertain healthcare service duration.
    
    \textbf{Page 5 (MT Department Service Planning):}
    List of variouse applications of compuational methods in Medical Technology Department. The applications not always were related to the scheduling.
    
    \textbf{Pages 5-6 (Conclusions):}
    The most common direction of research involves outpatient appointments, inpatient beds, and operating rooms. Then authors repeating the statements from the body of the paper. The last pharagraph highlights the importance of service scheduling in medical technology department. (reflaction: the conclusion is unclear.) 