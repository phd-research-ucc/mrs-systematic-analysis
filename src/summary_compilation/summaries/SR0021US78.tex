\section{ SR0021US78 }


\subsection{Meta}

    \textbf{Title:}
    Surgical demand scheduling: a review

    \begin{table}[H]
        \centering
        \begin{tabular}{|c|c|c|c|c|c|c|c|c|}
            \hline
                \textbf{Rank} & \textbf{Grasp} & \textbf{Grade} & \textbf{Type} & \textbf{Outcome} & \textbf{Domain} & \textbf{COV19} & \textbf{CoI} & \textbf{DB} \\
            \hline
                3 & 85\% & B & A & N & B & No & ?? & No \\
            \hline
        \end{tabular}
        \caption{Reference's metadata}
        \label{tab:SR0021US78}
    \end{table}

\subsection{Summary}
    James M. Magerlein and James B. Martin \cite{x235} conducted qualitative research in the area of surgical demand scheduling. The authors analysed the literature from 1968 to 1978 in the scope of advance scheduling with operating room time, advance scheduling with multiple constraints, allocation of scheduling procedures, and procedure time estimation. The paper is critical to poor implementation of the existing solutions and a broad gap between the proposed solutions and real-world requirements from these solutions. The work is well structured and full of discussions of the examples in the existing literature. The value of this review lies in introducing the Admissions Scheduling and Control System (ASCS), which needs further clarification. In addition, the paper gives a good understanding of the surgical demand scheduling before 1978.

\subsection{Notes}
    \begin{itemize}
        \item Adminssions Scheduling and Control System (ASCS);
        \item Poor implementation rate in 1978;
    \end{itemize}


\subsection{Reading}
    \textbf{Abstract:}
    This is a literature review on surgery demand scheduling prior of the day of the surgery and on the the day of the surgery. The existing approaches are discussed and critisise for failing to implement the majority of the solutions in real hospitals.
    
    \textbf{Objectives:}
    To address high meddical resouce cost, low healthcare facility utilisatation, and increasing demand for departments dependined on the surgical suite. 

    \textbf{Page 1-2:}
    From 1968 to 1978 the research in the field of surgical demand scheduling progressed drastically. The importance of the efficient hospital service demand scheduling lies in reducing hospital expanses and improving quality of health care. There are next solutions: (1) Sufficient number of resources; (2) Schedule veriaty of resourses including human-resources; (3) Predict surgery demand; (3) Reduce time spent on preparations before and between surgeries. To reach the best possible results all four aspects should be addressed together. 
    
    \textbf{Page 2 (Surgical Scheduling System):}
    \underline{Scheduling} = advance scheduling OR/AND resource allocation. There is simple advance scheduling that consideres only total OR available time, and there is more complex approach when multiple constraints are considered such as patient beds, nurses, surgeons, anaesthegiologists, and medical equipment.

    \textbf{Page 2-6 (Advance Scheduling with Operating-Room Time):}
    The scheduling horizon veries from the surgery specialy. There are blocked and nonblocked scheduling system. \underline{Nonblocked Systes} = FCFS (FIFO) + max fill + surgery duration related. \underline{Blocked Booking System} = assigment resouces to surgery team for time period T. Advantages: (1) reduce avg. LOS; (2) reducing waiting list; (3) increase a number of operations per theatre; (4) reducing a number of cancellations; (5) reducing conflicts amoung surgery teams; - however, there is no emporical evidance of these advantages. Disadvantages of the Blocked System: (I) unfilled core time; (II) increasing LOS for patients not scheduled in advanced; (III) increasing waiting time for urgent cases.

    \textbf{Page 6-9 (Advance Scheduling with Multiple Constraints):}
    Adminssios Scheduling and Control System effects in 10 hospitals: avg census increase from 88\% to 94.1\% + zero emergency turnewey + ~50\% of cases were adanced scheduled. ASCS does not consider OR capacity as a constraint. \underline{ASCS can be improved}. It could be more attention to multi-constrained scheduling problem in literature.
    
    \textbf{Page 9-12 (Allocating Scheduling Procedures):}
    Allocationg = specific room + start time. Poor number of studies in this field. The literature lacks of applicable solutiosn. The rest of the section discusses various solutions in the existing studies.

    \textbf{Page 12-14 (Procedure Time Evaluation):}
    Time estimation is criticul in reducing uncertainty in surgery case allocation. By 1978 there were only three estimation methods: surgeon estimations, OR schedule estimations, and averaging the durations from the previous medical records. The first two methods were used most often. Short procedures are easeier to estimate. There is not consensus on good estimation approach. 
    
    \textbf{Page 14 (Discussion):}
    In 1978 scheduling techniques were effective but the hospitals continued to use standart policies. Lack of practical implementations of the existing solutions: (1) not sutisfactory for medical personnels; (2) not integrated scheduling; (3) not applicable to real environments; (4) poor evalluation of the real expances and earnings. Some solutions are to complex for healthcare proffectionals.